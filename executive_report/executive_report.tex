% Main document
\documentclass[11pt, a4paper]{report}
\usepackage{xcolor}%for enabling colours in documents
\definecolor{unswred1}{RGB}{238,49,36}
\definecolor{unswred2}{RGB}{197,40,28}
\definecolor{unswred3}{RGB}{158,28,15}
\definecolor{unswred4}{RGB}{122,12,0}
\definecolor{unswblue1}{RGB}{0,174,239}
\definecolor{unswblue2}{RGB}{0,146,200}
\definecolor{unswblue3}{RGB}{0,118,163}
\definecolor{unswblue4}{RGB}{0,91,127}
\definecolor{unswyellow}{RGB}{255,212,0}
\usepackage{pslatex}%for using post-script fonts
\pdfpagewidth=210 true mm%fix for pslatex
\pdfpageheight=297 true mm%fix for pslatex
\usepackage[margin=2.54cm]{geometry}%for setting margins correctly
\usepackage[english]{babel}%for dealing neatly with umlautes
\usepackage{units}%for consistent use of units
\usepackage{url}%for type setting urls
\usepackage{hyperref}%for internal document links
\hypersetup{pdfborder=0 0 0}
\hypersetup{colorlinks=true}
\hypersetup{allcolors=unswred2}
\usepackage{graphicx}%for including graphics
\usepackage{array}%for better table control and commands
\setlength{\extrarowheight}{1pt}
\usepackage{verbatim}%for typesetting code
\usepackage{latexsym}%for a few extra symbols - alternatively use amssymb
\usepackage{sectsty}%for nice colourful headings
\chapterfont{\color{unswred2}}
\sectionfont{\color{unswred2}}
\subsectionfont{\color{unswred2}}
\subsubsectionfont{\color{unswred2}}
\usepackage{tocbibind}%for adding bibliography to toc
\usepackage{graphicx} %image wrapping
\usepackage{wrapfig}
\usepackage{listings}
\usepackage{caption}
\usepackage{subcaption}
\definecolor{codegreen}{rgb}{0,0.6,0}
\definecolor{codegray}{rgb}{0.5,0.5,0.5}
\definecolor{codepurple}{rgb}{0.58,0,0.82}
\definecolor{backcolour}{rgb}{0.95,0.95,0.92}

\lstdefinestyle{mystyle}{
    backgroundcolor=\color{backcolour},   
    commentstyle=\color{codegreen},
    keywordstyle=\color{magenta},
    numberstyle=\tiny\color{codegray},
    stringstyle=\color{codepurple},
    basicstyle=\ttfamily\normalsize,
    breakatwhitespace=false,         
    breaklines=true,                 
    captionpos=b,                    
    keepspaces=true,                 
    numbers=left,                    
    numbersep=5pt,                  
    showspaces=false,                
    showstringspaces=false,
    showtabs=false,                  
    tabsize=2
}

\lstset{style=mystyle}



\renewcommand{\baselinestretch}{1.1}%





\begin{document}
\begin{center}
    \large\textbf{\textcolor{unswred2}{Investigating Speaker Diarization on Code-Switching 
    Speech Executive Summary}}
    {\textcolor{unswred2}{Bronston Ashford z5146619}}
  \end{center}

\subsection*{Justification for Thesis}

Code-switching (CS) speech, the act of alternating between two or more languages 
in or between sentences, is a prevalent phenomenon in  most multilingual societies 
\cite[2]{sitaramSurveyCodeswitchedSpeech2020}. 
As the world becomes increasingly interconnected, speech technologies have an 
increasing demand to interface with these non-standard forms of speech. Among these 
technologies, speaker diarization (SD) systems, which identify 'who spoke when' 
in an audio clip, play a significant part.

\vspace*{10pt}
SD systems serve as an important component in the pipeline of various speech 
technologies, such as speech recognition or meeting transcription. Since the 
quality of upstream SD directly impacts the performance of the parent technologies, 
audio recordings containing CS speech may be at a disadvantage compared to monolingual 
speech \cite[20]{parkReviewSpeakerDiarization2021}. Therefore, mitigating errors 
introduced by CS in audio would contribute to  better accessibility and understanding 
in multilingual societies.

\vspace*{10pt}
However, developing robust SD systems that can effectively handle CS speech 
requires an understanding of how much and to what extent CS speech impacts these 
systems. Yet, due to a notable lack of research in this domain, the question 
remains unexplored. Furthermore, contributing to the lack of research is the absence of freely available 
SD datasets containing CS speech. The quality of research on diarizing CS speech 
is intrinsically tied to the quality of available datasets, the generation of 
which can be challenging and time-consuming. Therefore, the addition of a free 
CS dataset to the field stands to facilitate future research and contribute to 
the development of this domain. 

\vspace*{10pt}
This thesis intends to contribute by initiating 
an investigation into the impact of CS speech on SD systems with a specific 
language pair. It aims to offer initial insights into how much SD systems are 
influenced by CS speech and to identify which aspects of CS speech present the 
greatest challenges. In doing so, a new CS dataset fit for diarization will be 
developed and made freely accessible.

\vspace*{10pt}

\subsection*{Objectives}

\subsection*{Literature Review}
\subsection*{Preliminary Work}
\subsection*{Plans}

\section*{Meeting Log}
\bibliographystyle{IEEEtran}
\bibliography{pubs}
\end{document}
